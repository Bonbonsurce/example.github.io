\documentclass[14pt, letterpaper]{article}
\usepackage[utf8]{inputenc}
\usepackage{amsmath}
 \usepackage[usenames]{color}
\usepackage{colortbl}
\usepackage{geometry} \geometry{verbose,a4paper,tmargin=2cm,bmargin=2cm,lmargin=2.5cm,rmargin=1.5cm}
\usepackage{indentfirst}
\usepackage{graphicx}%Вставка картинок правильная

\usepackage{float}%"Плавающие" картинки

\usepackage{wrapfig}%Обтекание фигур (таблиц, картинок и прочего)

\usepackage[russian]{babel} 
\title{Kuznetcov}
\author{Cherehov}
\date{December 2022}

\begin{document}

\maketitle

\section{Билеты 7 - 12}

\newpage
\subsection{7 Билет: Преобразования условий задачи ЛП (линейного программирвоания)}


\begin{enumerate}
    \item 
    
    Преобразование неравенств в равенства.
    Неравенства любого типа (с знаками: \leq или \geq) можно преобразовать в равенства путем добавление в левую часть неравенств дополнительных переменных 
    
    \item  Преобразование свободных переменных в неотрицательыне.
    Свободную переменную $x_{j}$ (т.е. переменную, которая может принимать как отрицательные, так и положительные значения) можно представить как разность двух неотрицательных переменных, слудующим образом:
    $$ x_{j} = x^{+}_{j} - x^{-}_{j}$$

    Например, для $x_{j} = -5$ положим $ x^{+}_{j} = 0 и x^{-}_{j} = 5$. Причем $x^{+}_{j} и x^{-}_{j}$ - не отрицательны.

    \item Преобразование задачи максимизации в задачу минимизации
    Задача максимизации функции $f(x_{1}, x_{2} ... x_{n})$  эквивалентна задаче минимизации функции - $f(x_{1}, x_{2} ... x_{n})$ , поскольку при решении обеих задач предоставляется один и тот же набор переменных: $x_{1}, x_{2} ... x_{n}$.

    Канонический формат целевой функции: $z = cx \rightarrow max;$, чтобы нам привести целевую функцию к каноническому виду, следует поменять знаки и стремление функции, например: 
    $z = cx \reightarrow max$ эквивалентно $z = -cx \rightarrow min$

    Решив задачу, не забудьте восстановить знак!
    
    \item Противоположные неравенства в ограничениях  (фаналогично поступаем, если противоположный знак).
    $$\sum\limits_{j \in N} a_{ij}x_{j} \geq b_{i}$$
    также компенсируются неотрицательными дополнительными переменными:
    $$y_{i} = \sum\limits_{j \in N} a_{ij}x_{j} - b_{i} \geq 0 $$.
\end{enumerate}

\paragraph{Приведение к каноническому виду}

Каноническая задача ЛП имеет следующий вид:
$$z = cx \rightarrow max;\qquad ax = b;\qquad x \geq 0,$$

в которой все переменные неотрицательны, а отношения во всех ограничения - равенства. Для решения задачи ЛП ее следует привести именно к такой форме. ЭИо достигается путем вышеупомянутых преобразований.


\newpage
\subsection{\textcolor{red}{8 Билет: Графический метод решения задачи ЛП (линейного программирвоания)}}


    Графический метод наглядно иллюстрирует устройство задач ЛП и методов их решения, но не пригоден для решения практических задач.

Чтобы полнее погрузиться в содержание методов, рассмотрим очень простую задачу поиска наименьшего значения линейного функционала:
\begin{equation}
    z = 3x_{1} + 5x_{2} \rightarrow max
\end{equation}
    
С двумя неизвестными и ограничениями типа \leq

\begin{equation}
    2x_{1} + 2x_{2} \leq 20
\end{equation}

\begin{equation}
    x_{1} + 3x_{2} \leq 24 
\end{equation}

\begin{equation}
    2x_{1} + 2x_{2} \leq 16
\end{equation}

А также условиями неотрицательности:
\begin{equation}
    x_{1},x_{2} \geq 0
\end{equation}



Графический метод можно использовать для решения задач с произвольным количеством ограничений (2) - (4), в качестве которых могут быть любые нестрогие равенства. Определим $W$ - множество допустимых решений этой задачи.

$$ W = \{ x_{1},x_{2} \geq 0 \qquad | \qquad  2x_{1} + 2x_{2} \leq  20;\qquad  x_{1} + 3x_{2}  \leq 24;\qquad  2x_{1} + 2x_{2} \leq 16 \}  $$

Пошагово выполним решение задачи ЛП с использованием графического метода.
\begin{enumerate}
    \item \textcolor{blue}{Построение $W$.}
    
    Для решения задачи (1) - (5) первоначально графически построим множество допустимых решений - $W$.
    
При отображении декартовой системы координат, точки $(x_{1}, x_{2}) \in W$, будут находиться в первой четверти, а также ние и левее каждой из прямых:
$$L_{1}: 2x_{1} + 2x_{2} = 20,\qquad  L_{2}: x_{1} + 3x_{2} = 24,\qquad  L_{3}: 2x_{1} + 2x_{2} = 16$$
Прямые $L_{1}, L_{2}, L_{3}$ предстовляют ограничение задачи, построим их, используя, например, формулы уравнении в отрезках.


    Например, прямая $L_{1}$ пересекает ось абсцисс (условие $x_{2} = 0$) в точке с координатами ($10, 0$) и ось ординат (условие $x_{1} = 0$) в точке с координатами ($0, 10$). Построим отрезок с концами в указанных точках. 

    Точно также установим, что прямая $L_{2}$ проходит через точки с координатами ($24, 0$) и ($0, 8$), а прямая $L_{3}$ - через точки ($8, 0$) и ($0, 16$).

    Множество допустимых решений $W$ - пересечение областей первой четверти ниже и левее полученных отрезков.
    \begin{figure}[h]
    \centering       
    \includegraphics[width=0.4\linewidth]{grafmetod.png}        
    \end{figure}
        
    \item \textcolor{blue}{Поиск оптимального решения.}
    
    Для поиска оптимального решения исходной задачи, приравниваем значение целевой функции $= 3x_{1} + 5x_{2}$ к произвольному числу, например, $z = 60$ и получим \textcolor{blue}{линюю уровня} целевой функции (1). Все точки прямой: 
    $$L_{z} = 3x_{1} + 5x_{2} = z = 60 $$
    дают одиннаковое значение $z = 60$.

    При изменении $z$ линяя уровня $L_{z}$ смещается параллельно.
    Таким образом, для решения задачи остается, поглядев на рисунок (вот почему метод называется графическим!) найти наибольшее значение $z^{*}$, для которого $L_{z^{*}} \cap W \neq \emptyset$

    Координаты точки пересечения $(x_{1}^{*}, x_{2}^{*})$ и составляют оптимальное решение $(x^{*})$ задачи (1) - (5). Остается найти $x_{1}^{*}$ и $x_{2}^{*}$.
    
    \item \textcolor{blue}{Вычисление ответов.}

    Искомую точку пересечения $L_{z^{*}}$ и $W$ можно выбрать таким образом, чтобы она стала пересечением не менее, чем двух прямых из $L_{1}$, $L_{2}$ и $L_{3}$. Следовательно теперь остается выбрать подходящую пару уравнений и решить соответствующую систему линейных уравнений.
\end{enumerate}

\paragraph{\textcolor{blue}{Выводы.}}

Графический метод применим для поиска как наибольшего так и наименьшего значения линейной целевой функции и наглядно иллюстрирует все возможные случаи существования и отсутствия оптимального решения задачи ($W \neq \emptyset$, или $z^{*}$ не ограниченн).

Кроме того, графический метод указывает на ряд свойств полученного решения, которые будут введены далее:
\begin{enumerate}
    \item Множество $W$ независимо от числа и структуры ограничений задачи, обязательно является выпуклым многоугольником на плоскости , т. е. двухмерным выпуклым многогранником  

    Этот очевидный результат удается обобщить, заметив, что множество $W$ для любой задачи ЛП, независимо от количества и типа  ограничений и переменных - выпуклый многогранник, размерность которого зависит от количества переменных и количества ограничений типа равенства.
    
    \item Точку $B \in W$ назовем \textcolor{blue}{угловой} или \textcolor{blue}{крайней} точкой $W$, если отсутствует отрезок $[P, Q] \subset W$ середина которого точка $B$.
    $$ B = \frac{1}{2}P + Q$$

    Любой не пустой выпуклый многоугольник обязательно содержит \textcolor{blue}{угловые точки}, одна из которых (или не одна?), установим в дальнейшем, является $x^{*}$ - оптимальным решением задачи ЛП (если таковое существует).

    К примеру, множество $W$ рассматриваемой задачи содержит 5 точек:
    $$0(0, 0),\qquad A_{1}(0, 8),\qquad A_{2}(9, 0),\qquad A_{3}(4, 8),\qquad A_{4}(7, 5)$$
    Одна из них - $O$ - пересечение координатных осей, две последующие $A_{1}, A_{2}$ образованы пересечение координатной оси $OX$ и прямой $L_{1}$, аналогично $OY$ и прямой $L_{2}$ и остальные.
    
    \item Крайние (угловые) точки $W$ - сугубо геометрические понятия имеют вполне прозрачную алгебраическую и вычислительную структуру \textcolor{blue}{базисных решений}. Описание графического метода - подходящий повод для характеристики таких решений.

    Очевидно, что для независимого от коэффициентов целевой функции, какая-либо из перечисленных точек может быть оптимальным решением задачи ЛП среди конецного множества крайних точек многогранника $W$ существенно снижает вычислительную сложность и упрощает алгоритмы решения задачи ЛП.

    \item Количество угловых точек $W$ конеучно. Может случиться, что более двух если среди прямых, представляющих ограничения , окажется более двух, проходящих через одну точку (первый случай вырожденности задачи), имеются прямые, которые не пересекаются с $W$ (второй случай).

    Второй случай вырожденности условий задачи ЛП наступает, когда экстремум целевой функции достигается в нескольких угловых точках.
    
    \item Использование неравенств для описание ограничений не удобно, поэтому в будущем будем использовать описание ограничений через вектора.
\end{enumerate}

\paragraph{\textcolor{blue}{Комментарии.}}
\begin{enumerate}
    \item Графический метод можно применять для поиска как наибольшего, так и наименьшего значения целевой функции. Любое ограничение задачи может быть неравенствами любого знака.
    
    \item Нетрудно заметить, что при любом выборе параметров $c_{1}$ и $c_{2}$ целевой функции:
    $$z = c_{1}x_{1} + c_{2}x_{2} \rightarrow max, $$
    решением задачи можно выбрать одну из крайних (угловых) точек $ext(W)$. В дальнейшем установим, что это действительно так.
    
    \item Может случиться, что линия уровня $L_{z}$ параллельна одной из прямых $l_{i}$. В таком случае задача имеет много оптимальных решений, которые составляют две угловые (крайние) точки и весь граничный отрезок между ними.
    Назовем указанный случай \textcolor{blue}{первым типом вырожденности} задачи.
    \item Если $x^{*}$ ялвяется пересечением не двух, а большего количества граничных прямых, произвольно выберем пару подходящих прямых. Казанный случай назовем \textcolor{blue}{вторым типом вырожденности} задачи.
    \item Чертеж графического метода можно использовать для иллюстрации решения задачи ЛП симплексным методом
\end{enumerate}


\newpage
\subsection{9 Билет: Дифференцируемые функции нескольких переменных. Производная по направлению и градиент функции в точке}

\newpage
\subsection{10 Билет: Безусловная оптимизация функции нескольких переменных. Дважды дифференцируемые функции, 
положительная определенность матрицы. Достаточное условие положительной определенности. Примеры}

\newpage
\subsection{11 Билет: Определение задачи условной оптимизации. Множество допустимых решений и ограничения задачи 
услвоной оптимизации}

\newpage
\subsection{12 Билет: Выпуклые множества и их свойства}

\end{document}
