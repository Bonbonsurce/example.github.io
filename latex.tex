\documentclass[14pt, letterpaper]{article}
\usepackage[utf8]{inputenc}
\usepackage{amsmath}
 \usepackage[usenames]{color}
\usepackage{colortbl}
\usepackage{geometry} \geometry{verbose,a4paper,tmargin=2cm,bmargin=2cm,lmargin=2.5cm,rmargin=1.5cm}
\usepackage{indentfirst}
\usepackage{graphicx}%Вставка картинок правильная

\usepackage{float}%"Плавающие" картинки

\usepackage{wrapfig}%Обтекание фигур (таблиц, картинок и прочего)

\usepackage[russian]{babel} 
\title{Kuznetcov}
\author{Cherehov}
\date{December 2022}

\begin{document}

\maketitle

\section{Билеты 7 - 12}

\newpage
\subsection{7 Билет: Преобразования условий задачи ЛП (линейного программирвоания)}


\begin{enumerate}
    \item 
    
    Преобразование неравенств в равенства.
    Неравенства любого типа (с знаками: \leq или \geq) можно преобразовать в равенства путем добавление в левую часть неравенств дополнительных переменных 
    
    \item  Преобразование свободных переменных в неотрицательыне.
    Свободную переменную $x_{j}$ (т.е. переменную, которая может принимать как отрицательные, так и положительные значения) можно представить как разность двух неотрицательных переменных, слудующим образом:
    $$ x_{j} = x^{+}_{j} - x^{-}_{j}$$

    Например, для $x_{j} = -5$ положим $ x^{+}_{j} = 0 и x^{-}_{j} = 5$. Причем $x^{+}_{j} и x^{-}_{j}$ - не отрицательны.

    \item Преобразование задачи максимизации в задачу минимизации
    Задача максимизации функции $f(x_{1}, x_{2} ... x_{n})$  эквивалентна задаче минимизации функции - $f(x_{1}, x_{2} ... x_{n})$ , поскольку при решении обеих задач предоставляется один и тот же набор переменных: $x_{1}, x_{2} ... x_{n}$.

    Канонический формат целевой функции: $z = cx \rightarrow max;$, чтобы нам привести целевую функцию к каноническому виду, следует поменять знаки и стремление функции, например: 
    $z = cx \reightarrow max$ эквивалентно $z = -cx \rightarrow min$

    Решив задачу, не забудьте восстановить знак!
    
    \item Противоположные неравенства в ограничениях  (фаналогично поступаем, если противоположный знак).
    $$\sum\limits_{j \in N} a_{ij}x_{j} \geq b_{i}$$
    также компенсируются неотрицательными дополнительными переменными:
    $$y_{i} = \sum\limits_{j \in N} a_{ij}x_{j} - b_{i} \geq 0 $$.
\end{enumerate}

\paragraph{Приведение к каноническому виду}

Каноническая задача ЛП имеет следующий вид:
$$z = cx \rightarrow max;\qquad ax = b;\qquad x \geq 0,$$

в которой все переменные неотрицательны, а отношения во всех ограничения - равенства. Для решения задачи ЛП ее следует привести именно к такой форме. ЭИо достигается путем вышеупомянутых преобразований.


\newpage
\subsection{\textcolor{red}{8 Билет: Графический метод решения задачи ЛП (линейного программирвоания)}}


    Графический метод наглядно иллюстрирует устройство задач ЛП и методов их решения, но не пригоден для решения практических задач.

Чтобы полнее погрузиться в содержание методов, рассмотрим очень простую задачу поиска наименьшего значения линейного функционала:
\begin{equation}
    z = 3x_{1} + 5x_{2} \rightarrow max
\end{equation}
    
С двумя неизвестными и ограничениями типа \leq

\begin{equation}
    2x_{1} + 2x_{2} \leq 20
\end{equation}

\begin{equation}
    x_{1} + 3x_{2} \leq 24 
\end{equation}

\begin{equation}
    2x_{1} + 2x_{2} \leq 16
\end{equation}

А также условиями неотрицательности:
\begin{equation}
    x_{1},x_{2} \geq 0
\end{equation}



Графический метод можно использовать для решения задач с произвольным количеством ограничений (2) - (4), в качестве которых могут быть любые нестрогие равенства. Определим $W$ - множество допустимых решений этой задачи.

$$ W = \{ x_{1},x_{2} \geq 0 \qquad | \qquad  2x_{1} + 2x_{2} \leq  20;\qquad  x_{1} + 3x_{2}  \leq 24;\qquad  2x_{1} + 2x_{2} \leq 16 \}  $$

Пошагово выполним решение задачи ЛП с использованием графического метода.
\begin{enumerate}
    \item \textcolor{blue}{Построение $W$.}
    
    Для решения задачи (1) - (5) первоначально графически построим множество допустимых решений - $W$.
    
При отображении декартовой системы координат, точки $(x_{1}, x_{2}) \in W$, будут находиться в первой четверти, а также ние и левее каждой из прямых:
$$L_{1}: 2x_{1} + 2x_{2} = 20,\qquad  L_{2}: x_{1} + 3x_{2} = 24,\qquad  L_{3}: 2x_{1} + 2x_{2} = 16$$
Прямые $L_{1}, L_{2}, L_{3}$ предстовляют ограничение задачи, построим их, используя, например, формулы уравнении в отрезках.


    Например, прямая $L_{1}$ пересекает ось абсцисс (условие $x_{2} = 0$) в точке с координатами ($10, 0$) и ось ординат (условие $x_{1} = 0$) в точке с координатами ($0, 10$). Построим отрезок с концами в указанных точках. 

    Точно также установим, что прямая $L_{2}$ проходит через точки с координатами ($24, 0$) и ($0, 8$), а прямая $L_{3}$ - через точки ($8, 0$) и ($0, 16$).

    Множество допустимых решений $W$ - пересечение областей первой четверти ниже и левее полученных отрезков.
    \begin{figure}[h]
    \centering       
    \includegraphics[width=0.4\linewidth]{grafmetod.png}        
    \end{figure}
        
    \item \textcolor{blue}{Поиск оптимального решения.}
    
    Для поиска оптимального решения исходной задачи, приравниваем значение целевой функции $= 3x_{1} + 5x_{2}$ к произвольному числу, например, $z = 60$ и получим \textcolor{blue}{линюю уровня} целевой функции (1). Все точки прямой: 
    $$L_{z} = 3x_{1} + 5x_{2} = z = 60 $$
    дают одиннаковое значение $z = 60$.

    При изменении $z$ линяя уровня $L_{z}$ смещается параллельно.
    Таким образом, для решения задачи остается, поглядев на рисунок (вот почему метод называется графическим!) найти наибольшее значение $z^{*}$, для которого $L_{z^{*}} \cap W \neq \emptyset$

    Координаты точки пересечения $(x_{1}^{*}, x_{2}^{*})$ и составляют оптимальное решение $(x^{*})$ задачи (1) - (5). Остается найти $x_{1}^{*}$ и $x_{2}^{*}$.
    
    \item \textcolor{blue}{Вычисление ответов.}

    Искомую точку пересечения $L_{z^{ \cdot }}$ и $W$ можно выбрать таким образом, чтобы она стала пересечением не менее, чем двух прямых из $L_{1}$, $L_{2}$ и $L_{3}$. Следовательно теперь остается выбрать подходящую пару уравнений и решить соответствующую систему линейных уравнений.
\end{enumerate}

\paragraph{\textcolor{blue}{Выводы.}}

Графический метод применим для поиска как наибольшего так и наименьшего значения линейной целевой функции и наглядно иллюстрирует все возможные случаи существования и отсутствия оптимального решения задачи ($W \neq \emptyset$, или $z^{ \cdot }$ не ограниченн).

Кроме того, графический метод указывает на ряд свойств полученного решения, которые будут введены далее:
\begin{enumerate}
    \item Множество $W$ независимо от числа и структуры ограничений задачи, обязательно является выпуклым многоугольником на плоскости , т. е. двухмерным выпуклым многогранником  

    Этот очевидный результат удается обобщить, заметив, что множество $W$ для любой задачи ЛП, независимо от количества и типа  ограничений и переменных - выпуклый многогранник, размерность которого зависит от количества переменных и количества ограничений типа равенства.
    
    \item Точку $B \in W$ назовем \textcolor{blue}{угловой} или \textcolor{blue}{крайней} точкой $W$, если отсутствует отрезок $[P, Q] \subset W$ середина которого точка $B$.
    $$ B = \frac{1}{2}P + Q$$

    Любой не пустой выпуклый многоугольник обязательно содержит \textcolor{blue}{угловые точки}, одна из которых (или не одна?), установим в дальнейшем, является $x^{ \cdot }$ - оптимальным решением задачи ЛП (если таковое существует).

    К примеру, множество $W$ рассматриваемой задачи содержит 5 точек:
    $$0(0, 0),\qquad A_{1}(0, 8),\qquad A_{2}(9, 0),\qquad A_{3}(4, 8),\qquad A_{4}(7, 5)$$
    Одна из них - $O$ - пересечение координатных осей, две последующие $A_{1}, A_{2}$ образованы пересечение координатной оси $OX$ и прямой $L_{1}$, аналогично $OY$ и прямой $L_{2}$ и остальные.
    
    \item Крайние (угловые) точки $W$ - сугубо геометрические понятия имеют вполне прозрачную алгебраическую и вычислительную структуру \textcolor{blue}{базисных решений}. Описание графического метода - подходящий повод для характеристики таких решений.

    Очевидно, что для независимого от коэффициентов целевой функции, какая-либо из перечисленных точек может быть оптимальным решением задачи ЛП среди конецного множества крайних точек многогранника $W$ существенно снижает вычислительную сложность и упрощает алгоритмы решения задачи ЛП.

    \item Количество угловых точек $W$ конеучно. Может случиться, что более двух если среди прямых, представляющих ограничения , окажется более двух, проходящих через одну точку (первый случай вырожденности задачи), имеются прямые, которые не пересекаются с $W$ (второй случай).

    Второй случай вырожденности условий задачи ЛП наступает, когда экстремум целевой функции достигается в нескольких угловых точках.
    
    \item Использование неравенств для описание ограничений не удобно, поэтому в будущем будем использовать описание ограничений через вектора.
\end{enumerate}

\paragraph{\textcolor{blue}{Комментарии.}}
\begin{enumerate}
    \item Графический метод можно применять для поиска как наибольшего, так и наименьшего значения целевой функции. Любое ограничение задачи может быть неравенствами любого знака.
    
    \item Нетрудно заметить, что при любом выборе параметров $c_{1}$ и $c_{2}$ целевой функции:
    $$z = c_{1}x_{1} + c_{2}x_{2} \rightarrow max, $$
    решением задачи можно выбрать одну из крайних (угловых) точек $ext(W)$. В дальнейшем установим, что это действительно так.
    
    \item Может случиться, что линия уровня $L_{z}$ параллельна одной из прямых $l_{i}$. В таком случае задача имеет много оптимальных решений, которые составляют две угловые (крайние) точки и весь граничный отрезок между ними.
    Назовем указанный случай \textcolor{blue}{первым типом вырожденности} задачи.
    \item Если $x^{ \cdot }$ ялвяется пересечением не двух, а большего количества граничных прямых, произвольно выберем пару подходящих прямых. Казанный случай назовем \textcolor{blue}{вторым типом вырожденности} задачи.
    \item Чертеж графического метода можно использовать для иллюстрации решения задачи ЛП симплексным методом
\end{enumerate}


\newpage
\subsection{9 Билет: Дифференцируемые функции нескольких переменных. Производная по направлению и градиент функции в точке}

Функция двух переменных обычно записывается как $z = f(x; y),$ при этом переменные $x, y$ нызываются независимыми переменными или аргументами.

\paragraph{}
С геометрической точки зрения функция двух переменных $z = f(x; y)$ чаще всего представляет собой \textbf{поверхность трёхмерного пространства} (плоскость, цилиндр, шар, параболоид, гиперболоид и т. д.)

\paragraph{}
Это обычное дифференцирование функций от нескольких переменных, не более. Геометрически \textbf{функция двух переменных} чаще всего представляет собой \textbf{поверхность}, и значения "зет" у нас чётко ассоциируются с высотой. Таким образом, с позиции геометрии скорость изменения данной функции  - есть скорость изменения высоты. При этом совершенно понятно, что "негоризонтальная" поверхность изменчива - в каких-то направлениях она крута, в каких-то полога, а где-то таки "равнина". И \textbf{производная по направлению} как раз призвана охарактеризовать "ландшафт местности" (скорость изменения функции) в различных точках по различным направлениям.


\paragraph{}
Встанем в некоторую точку $M_{0}(x_{0}; y_{0})$ \textbf{области определения}. В зависимости от выбора точки нам доступен бесконечно малый "шажок" в некоторых или, что вероятнее, во всех направлениях. Направление традиционно обозначается исходящим из точки $M_{0}$ лучом $l$, лежащим в плоскости $XOY$. Сам луч можно определить с помощью угла (между ним и осью $OX$ либо $OY$), а еще лучше - с помощью \textbf{вектора}.

\paragraph{}
Если в точке $M_{0}(x_{0}; y_{0})$ существует производная по направлению луча $l$ (исходящего из точки $M_{0}$ и лежащего в плоскости $XOY$), то её можно рассчитать по следующей формуле:
$$\frac{\partial z}{\partial l} = z\prime_{x}(x_{0}; y_{0}) \cdot cos\alpha + z\prime_{y}(x_{0}; y_{0}) \cdot cos\beta, $$
где:

$z\prime_{x}(x_{0}; y_{0}), z\prime_{y}(x_{0}; y_{0})$  - \textbf{\textcolor{blue}{частные производные 1-го порядка}} в точке $M_{0}$;
$cos\alpha, cos\beta$ - \textbf{\textcolor{blue}{направляющие косинусы}} (координаты  \textbf{\textcolor{blue}{вектора}} единичной длины), однозначно определяющие данное направление.

\paragraph{}

Плоскость, проходящую через луч $l$ перпендикулярно плоскости $XOY$. Данная плоскость "высекает" из поверхности $z = f(x; y)$ Пространственную линию $L$, которой, очевидно, принадлежит точка: $$M(x_{0}, y_{0}, z_{0})\qquad (z_{0} = f(x_{0}; y_{0})).$$ Производная по направлению численно равна тангенсу угла $\varphi$ \textbf{между} \underline{касательной} к линии $L$ в точке $M$ и \underline{плоскостью} $XOY: \frac{\partial z}{\partial l} = tf\varphi, $ где $\varphi$- это угол \textbf{между} касательной к линии $L$ в точке $M$ и ее ортогональной проекцией на плоскость $XOY$, т. е. направлением луча $l$)

\paragraph{}

\textbf{Градиентом функции} $z = f(x; y)$ \textbf{в точке} $M(x_{0}, y_{0}$ называется \underline{направленный отрезок} 
    $$grad \: z(M_{0}) = z\prime_{x}(M_{0}) \cdot \bar i + z\prime_{y}(M_{0}) \cdot \bar j, $$
    отложенный от точки $M_{0}$, который показывает \textbf{направление и скорость} наискорейшего роста функции $z = f(x; y)$ в \underline{данной точке}.

Производная по некотору направлению $l$ в точке $M_{0}$ - это \textbf{проекция градиента} в данной точке на данное направление
$$\frac{\partial z}{\partial l_{|M_{0}}} = |grad \: z(M_{0})| \cdot cos\phi, \: $$
где $|grad \: z(M_{0})|$ - длина градиаента;

$\phi = \angle(grad \:z(M_{0}); l)$ - угол между градиаентом и данным направлением.

\paragraph{}
В свою очередь из этой формулы следует, что производная по направлению достигает максимального значения при $cos\phi = 1$, то есть когда:
    $$\phi = \angle(grad \:z(M_{0}); l) = 0$$- 
    направление $l$ совпадает с направлением градиента.

\paragraph{}
То же самое и с 3 переменными, просто добавляется одно измерение и одно слагаемое.
Если в точке $M_{0}(x_{0}, y_{0}, z_{0})$ существует производная по направлению пространственного луча $l$ (исходящего из точки $M_{0}$), то её можно рассчитать по следующей формуле:
    $$\frac{\partial u}{\partial l} = u\prime_{x}(x_{0}; y_{0}; z_{0}) \cdot cos\alpha + u\prime_{y}(x_{0}; y_{0}; z_{0}) \cdot cos\beta + u\prime_{z}(x_{0}; y_{0}; z_{0}) \cdot cos\gamma, $$

\newpage
\subsection{10 Билет: Безусловная оптимизация функции нескольких переменных. Дважды дифференцируемые функции, 
положительная определенность матрицы. Достаточное условие положительной определенности. Примеры}

\paragraph{}
Задача нахождения минимума или максимума функции $f(X), X$ - вектор n-мерного евклидового пространства; $x_{1}, x_{2} \idots x_{n}$ - компоненты вектора X. Обычно эта задача записывается как: 
    $$f(X) \rigtharrow min(max), \qquad X \in R^{n}$$
\paragraph{}
\textbf{Экстремумом функции двух переменных} называется её максимальное или минимальное значение на заданном множестве изменения переменных. Точка $M_{0} (x_{0}, y_{0})$ называется \textbf{точкой максимума (минимума) - $z_{max} (z_{min})$} функции $z = f(x, y)$, если существует окрестность точки $M_{0}$, такая, что для всех точек $(x, y)$ из этой окрестности выполняется неравенство:
    $$f(x_{0}, y_{0}) \geq f(x, y) \qquad (f(x_{0}, y_{0}) \leq f(x, y)).$$

    \begin{figure}[h]
    \centering       
    \includegraphics[width=0.6\linewidth]{10ti.png}        
    \end{figure}

\paragraph{}
Наибольшая величина из локальных максимумов называется \textbf{глобальным максимумом}, наименьший из локальных минимумов - \textbf{глобальным минимумом}.

\paragraph{}

\textbf{Линией уровня функции} $U + F(X, Y)$ называется линия $F(X, Y) = C$ на плоскости $XOY$, в точках которой функция сохраняет постоянное значение $U = C$.
Хорошо известны примеры линий уровня - уровня одиннаковых высот на топографической карте и линии одинакового баромитрического давления карте погоды.
\paragraph{}

\textbf{Градиентом функции} $u = f(x; y)$ в точке $M_{0}(x_{0}, y_{0})$ называется вектор, координаты которого равны значениям частных производных функции в точке $M_{0}$:
    $$grad \overline{f(M_{0})} = \{ \}$$

\newpage
\subsection{11 Билет: Определение задачи условной оптимизации. Множество допустимых решений и ограничения задачи 
услвоной оптимизации}

\newpage
\subsection{12 Билет: Выпуклые множества и их свойства}
\paragraph{}
Множество $W \subset R^{n}$ назовём \textbf{выпуклым}, если с любыми двумя точками $x, y \in W$ это множество содержит $[x, y] \subset W$ - весь отрезок с концами в этих точках.

\paragraph{}
Примеры выпуклыx множеств:
\begin{enumerate}
    \item одна точка, 
    \item все пространство $R^{n}$,
    \item гиперплоскость,
    \item отрезок $[x, y]$,
    \item треугольник или шар $R^{n}$.
\end{enumerate}

Невыпуклые:
\begin{enumerate}
    \item окружность,
    \item звезда,
    \item фасолина, 
    \item сосиска.
\end{enumerate}

\paragraph{}
\textbf{Утверждение 1.} Пересечение любой совокупности выпуклых множеств является выпуклым множеством.

\textbf{Утверждение 2.} Выпуклым множеством является множество $W$, заданное набором линейных ограничений:
%формула без пометки номера
\begin{displaymath}
   	\begin{split}
   		\begin{aligned}
   			W = \{ x \in R^{n}\: | \: a[M_{1},N]x[N] \leq b[M_{1}], \\ a[M_{2},N]x[N] \geq b[M_{2}], \qquad a[M_{3},N]x[N] = b[M_{3}] \}
   		\end{aligned}
   	\end{split}
   \end{displaymath}
\paragraph{}
Таким образом, множество допустимых решений любой задачи линейного программирования является выпуклым.

\paragraph{}
Выпуклость множества допустимых решений $W$ существенно расширяет возможности рпешения задач оптимизации
    $$P: \qquad \f(x) \rightarrow max(min), \qquad x \in W.$$

Поэтому в ряде случаев множество $W$ в условиях задачи заменяют большим $W \subset W\prime$ - выпуклым. В связи с этим введем определение. \textbf{Выпуклой оболочкой} произвольного множества $W \subset R^{n}$ назовем наименьшее $W\prime \subset R^{n}$ - выпуклое множество, содержащее $W$.

\paragraph{}

\textbf{Утверждение 3.} Предоставленное определение корректно, любое множество $W\prime \subset R^{n}$ имеет выпуклую оболочку.

\textbf{Утверждение 4. (Теорема Хелли).} Если любая тройка среди $n$ выпуклых множеств на плоскости имеет не пустое пересечение, то и все $n$ множеств также имеют не пустое пересечение.
\paragraph{}
Важными свойствами выпуклых множеств является отделимость, возможность разделить неперемекающиеся выпуклые множества гиперплоскостями, исключая возможность "перемешивания" составляющих их точек.
\paragraph{}
Другое важное свойство характеризует структуру выпуклого множества. Выпуклым является множество, определяемое набором крайних (или угловых) точек и лучей крайних направлений. Все остальные точки выпуклого множества определяются как выпуклые или конические комбинации перечисленных соответственно.
\paragraph{}
Такая "ограниченная" структура выпуклого множества позволяет существенно сократить область поиска оптимального решения задачи оптимизации, тем самым повышая эффективность используемых алгоритмов.


\newpage
\subsection{13 Билет: Теоремы отделимости выпуклых множеств}

Важными свойствами выпуклых множеств является отедлимость, возможность разделить непересекающиеся выпуклые множества гиперплоскостями, исключая возможность "перемешивания" составляющих их точек

Другое важное свойство характеризует структуру выпуклого множества. Выпуклым является множество, определяемое набором крайних (или угловых) точек и лучей крайних направлений. Все остальные точки выпуклого множества определяются как выпуклые или конические комбинации перечисленных соответственно

Такая "ограниченная" структура выпуклого множества позволяет существенно сократить область поиска оптимального решения задачи оптимизации, тем самым повышая эффективность используемых алгоритмов.

\textbf{Теорема 1 (О проекции точки на выпуклое множество).}

Пусть $W$ - выпуклое замкнутое множество, $v \notin W$. Тогда существует удинственная точка $p = pw(v) \in W$ - проекция $v$ на $W$.

\textbf{Отделимость выпуклого множества от внешней точки}


\textbf{Теорема 2.}

Пусть $W$ - выпуклое замкнутое множество, $v \notin W$. Тогда существует гиперплоскость
    $$ E = E(c, b): \qquad  cv = b,$$
и $cs \lt$ для всех $x \in W$ (то есть $W \subset V^{-}_{R}$)

Теорема утверждает, что для любой точки $v$ вне выпуклого замкнутого множества $W$, найдется гиперплоскость, которая разделяет $R^{n}$ на полупространства в одном из которых находится $W$, а в другом точка $v$.

\textbf{Теорема 3 (об опорной гиперплоскости).}

Пусть $v \in Fr(W)$ - граничная точка выпуклого замкнутого множества $W$. Тогда существует опорная гиперплоскость:
$$E = E(c, b): \qquad cx = b,$$
такая, что

$$cx \leq b \qquad для всех \qquad x \in W$$

то есть $W \subset V^{-}_{E} \cup E.$

Теорема фактически устанавливает, что для любой граничной точки $v \in Fr(W)$ выпуклого множества $W$ найдется гиперплоскость, проходящая через $v$, одно из полупространств в которой содержит все множество $W$.
\end{document}



